\chapter*{Введение} % * не проставляет номер
\addcontentsline{toc}{chapter}{Введение} % вносим в содержание

Эффективная локализация транспортных средств, в частности грузовых автомобилей на пересеченной местности, представляет собой сложную задачу, требующую использования современных технологий. 

Пересеченная местность включает в себя разнообразные природные условия, например, смешанный лес, и характеризуется наличием 20\% и более различного рода препятствий, затрудняющих передвижение. Эти препятствия могут быть как естественного происхождения (деревья, склоны, реки), так и искусственно созданными объектами (заборы, рельсовые пути), что значительно усложняет процесс обнаружения и идентификации объектов. 
Одним из перспективных подходов к решению данной задачи является анализ оптических и инфракрасных снимков. 

Инфракрасные снимки, полученные с тепловизоров, позволяют обнаруживать объекты на основе их тепловых характеристик. Оптические снимки, полученные с помощью видеокамер, дают четкое визуальное представление об объектах на местности. Эти данные хорошо дополняют тепловизионные, обеспечивая многосторонний анализ.

Для размещения видеокамер и тепловизоров используются беспилотные летательные аппараты. Они обеспечивают высокую мобильность, позволяя эффективно обследовать труднодоступные зоны и получать данные в режиме реального времени. 

Методы машинного обучения являются важным инструментом для обработки и анализа данных, полученных с видеокамер и тепловизоров. Современные алгоритмы машинного обучения позволяют эффективно совмещать и интерпретировать данные, выявляя ключевые признаки, указывающие на наличие грузовых автомобилей на исследуемой территории. Это позволяет создавать интеллектуальные системы, способные автоматически обнаруживать и локализовать транспортные средства с высокой точностью.

Разработка алгоритма локализации грузовых автомобилей на пересеченной местности имеет критическое значение для обеспечения безопасности на охраняемых территориях. Предлагаемая интеллектуальная система обнаружения транспортных средств позволит своевременно выявлять грузовые автомобили, проникающие на охраняемые территории. Данный подход будет способствовать предотвращению несанкционированного проникновения, повысит уровень безопасности объектов и территорий, находящихся под охраной.