%% Не менять - Do not modify
%%\input{my_folder/summary_settings} 
\chapter*[Count-me]{Реферат} % * - не нумеруем
\thispagestyle{empty}% удаляем параметры страницы
%% Для того, чтобы значения счетчиков корректно отобразились, необходимо скомпилировать файл 2-3 раза
На \total{mypages}~c.,  
\formbytotal{myfigures}{рисун}{ок}{ка}{ков},
\formbytotal{mytables}{таблиц}{у}{ы}{},
\formbytotal{myappendices}{приложен}{ие}{ия}{ий}%.  

%\noindent
{\MakeUppercase{Ключевые слова: \keywordsRu}.} % Ключевые слова из renames.tex

Тема выпускной квалификационной работы: <<\thesisTitle>>

\abstractRu % Аннотация из renames.tex

%\abstractRu % УДАЛИТЬ. Повтор иллюстрации переноса текста на вторую страницу



\printTheAbstract % не удалять
\total{mypages}~pages, 
\total{myfigures}~figures, 
\total{mytables}~tables,
\total{myappendices}~appendices%.

%\noindent
{\MakeUppercase{Keywords: \keywordsEn}.} % Ключевые слова из renames.tex 
	
The subject of the graduate qualification work is <<\thesisTitleEn>>.
	
	
\abstractEn % Аннотация из renames.tex

%\abstractEn % УДАЛИТЬ. Повтор для иллюстрации переноса текста на вторую страницу
	


%% Не менять - Do not modify
\thispagestyle{empty}
%\setcounter{sumPageLast}{\value{page}} %сохранили номер последней страницы Задания
%\setcounter{sumPages}{\value{sumPageLast}-\value{sumPageFirst}}
%sumPageLast \arabic{sumPageLast}
%
%sumPages \arabic{sumPages}
%\restoregeometry % восстанавливаем настройки страницы
%\input{my_folder/summary_settings_restore}	% настройки - конец