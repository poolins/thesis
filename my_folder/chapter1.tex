\chapter{Обзор предметной области} \label{ch1}

% не рекомендуется использовать отдельную section <<введение>> после лета 2020 года
%\section{Введение. Сложносоставное название первого параграфа первой главы для~демонстрации переноса слов в содержании} \label{ch1:intro}

Хорошим стилем является наличие введения к главе, которое \textit{начинается непосредственно после названия главы, без оформления в виде отдельного параграфа}. Во введении может быть описана цель написания главы, а также приведена краткая структура главы. Например, в параграфе \ref{ch1:sec1} приведены примеры оформления одиночных формул, рисунков и таблицы. Параграф \ref{ch1:sec2} посвящён многострочным формулам и сложносоставным рисункам.

Текст данной главы призван привести \textit{краткие} примеры оформления текстово-графических объектов. Более подробные примеры можно посмотреть в следующей главе, а также в рекомендациях студентам \cite{spbpu-student-thesis-template-author-guide}. 


\section{Задача обеспечения безопасности} \label{ch1:sec1}

В современных условиях резкого осложнения криминогенной обстановки, роста числа террористических и диверсионных актов проблема обеспечения безопасности объектов входит в разряд приоритетных задач, как для государственных организаций, так и для организаций любой другой формы собственности и для собственников любых видов недвижимости.

По субъектам организации охранной деятельности различаются:
\begin{itemize}
    \item государственная охрана, представляющая собой специализированные автономные организационные структуры, предназначенные для охраны объектов особой государственной важности, перечень которых устанавливается специальными нормативными актам правительства;
    \item ведомственная охрана, представляет собой специализированные, вооруженные (как правило) подразделения, осуществляющие охрану различных объектов, входящих в структуру определенного ведомства;
    \item вневедомственная охрана это специализированные подразделения, осуществляющие охрану объектов, принадлежащих различным ведомствам и частным лицам, на контрактной, возмездной основе.
\end{itemize}

Формы организации, номенклатура охранных услуг, методы и средства реализации охранной деятельности в основном определяются тем, какому субъекту охранной деятельности подведомственен данный объект, кем он охраняется - государством, ведомством, вневедомственной государственной или частной охраной.

На подступах к объектам охраны создаются активные и пассивные защитные препятствия, например: система физических препятствий (инженерные заграждения), специальное оборудование мест хранения секретных документов, контрразведывательное обеспечение.

Надежность охраны достигается детальным построением системы охраны, правильной организацией и бдительным несением службы нарядами.

В зависимости от местности, характера и категории объекта и других особенностей охрана объектов может быть организована следующими способами:
\begin{enumerate}
    \item По периметру. Технические средства охраны выставляются на границе охраняемой территории и преграждают доступ к объекту вне пропускных пунктов (именно таким способом, как правило, охраняются некоторые режимные объекты).
    \item По отдельным объектам. Личный состав выставляется непосредственно на охраняемом объекте (примером такого способа охраны может быть порядок организации охраны складов МВД по хранению боеприпасов).
    \item Смешанным способом. По периметру и отдельным объектам одновременно.
    \item Способом оперативного дежурства. Охранные функции осуществляются комплексом инженерно-технических средств охраны при дежурном состоянии сил охраны (примером такого способа охраны является организация охраны любой атомной электростанции).
\end{enumerate}

На основе изученных статистических данных, можно сделать вывод, что охраняемые объекты наиболее часто подвергаются следующим видам угроз:
\begin{itemize}
    \item несанкционированное проникновение на территорию;
    \item несанкционированное получение информации об объекте или иной закрытой информации путем установки на объекте скрытых средств негласного получения информации;
    \item нападение на охраняемый объект с целью хищения материальных ценностей;
    \item угрозы жизни и здоровью персонала и посетителей объекта, в том числе взятие заложников с целью достижения иных целей;
    \item нарушение инфраструктуры и линий жизнеобеспечения объекта охраны;
    \item нарушение режима работы объекта, с целью прекращения его функционирования;
    \item саботаж технических средств охраны.
\end{itemize}


\section{Существующие методы обеспечения безопасности} \label{ch1:sec2} 

Методы обеспечения безопасности объектов включают в себя широкий спектр мер, направленных на предотвращение и минимизацию рисков, связанных с несанкционированным доступом, диверсиями, террористическими актами и другими угрозами. Рассмотрим основные существующие методы обеспечения безопасности.

\subsection{Физическая охрана}
Физическая охрана объектов заключается в использовании специально обученного персонала для защиты территории и имущества от различных угроз. Может быть реализована различными способами.
\begin{enumerate}
    \item Патрулирование территории. \\
    Сотрудники охраны проводят регулярные обходы по территории объекта с целью выявления и предотвращения возможных угроз. Патрулирование может осуществляться как пешком, так и на транспортных средствах.
    \item Посты охраны. \\
    Размещение стационарных постов охраны на ключевых точках объекта, таких как въезды и выезды, зоны повышенного риска или наиболее уязвимые участки периметра. 
    \item Контрольно-пропускные пункты (КПП). \\
    Организация пропускного режима с использованием систем идентификации (карты доступа, биометрические данные), журналов регистрации посетителей и персонала, а также физического контроля со стороны сотрудников охраны.
\end{enumerate}
\subsection{Технические средства охраны}
Современные технологии позволяют существенно повысить уровень безопасности объектов. Технические средства охраны включают в себя следующее.
\begin{enumerate}
    \item Системы видеонаблюдения. \\
    Установка камер видеонаблюдения по периметру и внутри объекта для круглосуточного мониторинга и записи событий. Современные системы видеонаблюдения могут оснащаться функциями распознавания лиц, анализа поведения и автоматической сигнализации при выявлении подозрительных действий.
    \item Системы контроля доступа. \\
    Использование электронных замков, турникетов, шлюзов и других устройств для регулирования доступа на территорию объекта. Данные системы интегрируются с базами данных сотрудников и посетителей, обеспечивая индивидуальные уровни доступа и фиксируя все попытки входа и выхода.
    \item Системы сигнализации. \\
    Установка охранных, пожарных и тревожных сигнализаций, которые автоматически оповещают службу безопасности и экстренные службы о возникновении угрозы. Сигнализация может быть оснащена датчиками движения, разбития стекла, дыма, газа и других параметров.
\end{enumerate}
\subsection{Инженерно-технические средства охраны}
В современных условиях для обеспечения охраны территорий используются разнообразные инженерно-технические средства. Эти средства могут включать как естественные, так и искусственные барьеры, которые препятствуют незаконному проникновению на охраняемую территорию. Основное внимание уделяется искусственным заградительным сооружениям, которые обеспечивают физическую защиту периметра объекта, элементов зданий и помещений от несанкционированного доступа.
\begin{enumerate}
    \item Заграждения и противотаранные устройства.
    \begin{enumerate}
        \item Колючая проволока и армированная колючая лента. Используются для создания заграждений, которые затрудняют или делают невозможным преодоление препятствия. АКЛ обладает высокой прочностью, упругостью и стойкостью к коррозии благодаря оцинкованному покрытию. Этот тип заграждения является одним из самых распространенных и недорогих средств защиты.
        \item Сварные сетчатые панели. Применяются для ограждения промышленных объектов, объектов городской инфраструктуры и частной собственности. Конструкции таких заграждений могут включать различные дополнительные технические средства обнаружения, имеют минимальные сроки монтажа и хорошо вписываются в городскую инфраструктуру.
        \item Просечная вытяжная сетка. Обеспечивает устойчивость к ветровым нагрузкам и может быть выполнена из различных материалов, таких как низкоуглеродистая сталь, алюминий или оцинкованная сталь.
        \item Сварные трубы и радиопрозрачные заграждения. Сварные трубы часто используются для создания прочных ограждений, покрытых полимерными материалами. Радиопрозрачные заграждения, выполненные из пластика и стеклопластика, предназначены для защиты радиотехнических комплексов, так как они не препятствуют приему и передаче электромагнитных волн.
        \item Электрошоковые заграждения. Используются для создания высокоэффективных барьеров с применением безопасного электрошокового воздействия. Эти заграждения питаются от напряжения 220В и вызывают болезненные ощущения, вынуждая злоумышленника отказаться от противоправных действий.
        \item Железобетонные противотаранные заграждения. Обеспечивают надежную защиту от таранных атак. Внутри железобетонных плит могут прокладываться кабели для систем сигнализации и видеонаблюдения.
    \end{enumerate}
    \item Средства регулирования доступа.
    \begin{enumerate}
        \item Шлагбаумы. Используются для контроля въезда и выезда автотранспорта на охраняемую территорию. Управление шлагбаумами может осуществляться с пульта охраны, пульта-брелока водителя или с помощью бесконтактных карт и жетонов.
        \item Ворота. Существуют различные типы ворот, такие как распашные, откатные и консольные. Они могут быть оснащены датчиками контроля положения, электроприводами и дополнительными заградительными элементами.
        \item Противотаранные устройства и блокираторы. Включают мобильные блоки, выдвижные столбы и стационарные дорожные блокираторы, предназначенные для предотвращения несанкционированного проезда автотранспорта. Некоторые устройства могут оснащаться датчиками для обнаружения ударов или вибраций.
        \item Дорожные шипы и козырьковые заграждения. Дорожные шипы предназначены для принудительной остановки автотранспорта, пробивая шины. Козырьковые заграждения устанавливаются на верхней части ограждений для предотвращения перелазов.
    \end{enumerate}
\end{enumerate}

\section{Беспилотные летательные аппараты}
Беспилотные летательные аппараты (БПЛА) представляют собой летательные устройства, способные перемещаться в воздушном пространстве без участия пилота на борту. Они могут управляться дистанционно или посредством установленного на борту автономного программного обеспечения. БПЛА не нуждаются в использовании аэродрома или посадочной площадки и способны взлетать в любой географической точке, что делает их особенно удобными для использования на пересеченной местности.

БПЛА классифицируются в зависимости от конструкции, что напрямую влияет на их летные характеристики. Наиболее популярными моделями БПЛА, которые могут быть использованы на пересеченной местности, являются квадрокоптеры, мультикоптеры с фиксированным крылом и беспилотные авиационные системы вертолетного типа. 

Квадрокоптеры характеризуются малыми размерами и весом, высокой маневренностью и скоростью, возможностью вертикального взлета и посадки, что делает их идеальными для использования на небольших площадях и в труднодоступных местах.

Мультикоптеры с фиксированным крылом обладают большей грузоподъемностью и дальностью полета, что позволяет им выполнять более сложные задачи, такие как перевозка грузов или проведение длительных наблюдений. Они обеспечивают более стабильную и плавную полетную динамику, что улучшает качество съемки.

БПЛА вертолетного типа, благодаря подъемной силе винтов-роторов, отличаются высокой маневренностью и способностью зависать в воздухе для тщательного исследования объекта. Однако их относительная малая скорость полета и ограниченное время нахождения в воздушном пространстве снижают их радиус действия.

Применение БПЛА на пересеченной местности имеет ряд преимуществ по сравнению с пилотируемыми летательными аппаратами или космической съемкой.

\begin{enumerate}
    \item Высокая мобильность. \\
БПЛА могут легко перемещаться по пересеченной местности и достигать труднодоступных зон, что обеспечивает широкий охват территорий.
\item Быстрое развертывание. \\
БПЛА можно оперативно развернуть и использовать практически в любой точке без необходимости в специальной инфраструктуре, такой как аэродромы.
\item Многофункциональность сенсоров. \\
Возможность оснащения различными камерами и сенсорами, включая оптические и инфракрасные, что позволяет получать детализированные и разнообразные данные.
\item Оперативность получения данных. \\
БПЛА позволяют получать данные в режиме реального времени, что ускоряет процесс анализа и принятия решений.
\item Низкие эксплуатационные расходы. \\
По сравнению с пилотируемыми летательными аппаратами, БПЛА имеют меньшие эксплуатационные расходы, особенно при съемке небольших и средних территорий.
\item Безопасность операторов. \\
Операторы могут управлять БПЛА на безопасном расстоянии от потенциально опасных или труднодоступных зон, снижая риск для человеческой жизни.
\end{enumerate}

Недостатки использования БПЛА для локализации грузовых автомобилей на пересеченной местности. 
\begin{enumerate}
    \item Ограниченное время полета. \\
Большинство БПЛА имеют ограниченное время полета из-за емкости аккумуляторов, что может требовать частой замены батарей или использования нескольких аппаратов для покрытия больших территорий.
\item Чувствительность к погодным условиям. \\
Резкое ухудшение погодных условий, такие как сильный ветер, дождь или снег, могут существенно влиять на летные характеристики и стабильность работы БПЛА.
\item Ограниченная грузоподъемность. \\
Возможности по установке тяжелых сенсоров или дополнительного оборудования ограничены из-за небольшой грузоподъемности большинства моделей.
\item Правовые ограничения. \\
В некоторых регионах существуют строгие правила и ограничения на использование БПЛА, что может затруднить их применение.
\end{enumerate}


\section{Выводы} \label{ch1:conclusion}

Текст выводов по главе \thechapter.

Кроме названия параграфа <<выводы>> можно использовать (единообразно по всем главам) следующие подходы к именованию последних разделов с результатами по главам:
\begin{itemize}
	\item <<выводы по главе N>>, где N --- номер соответствующей главы;
	\item <<резюме>>;
	\item <<резюме по главе N>>, где N --- номер соответствующей главы.
\end{itemize}

Параграф с изложением выводов по главе \textit{является обязательным}.

%% Вспомогательные команды - Additional commands
%
%\newpage % принудительное начало с новой страницы, использовать только в конце раздела
%\clearpage % осуществляется пакетом <<placeins>> в пределах секций
%\newpage\leavevmode\thispagestyle{empty}\newpage % 100 % начало новой страницы